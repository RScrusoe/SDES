\documentclass[12pt,a4paper]{beamer}


% Preamble

\usepackage{graphics}
\usepackage{graphicx}
\usepackage{amsmath, amssymb}

\usepackage{listings}
\usepackage{color}
\usepackage{longtable}

\renewcommand{\int}{\mathbb{Z}}
\newcommand{\sqrtntwo}{\sqrt[n]{2}}

\begin{document}

\tableofcontents

\includeonly{file}

\chapter{First chapter}

Excel any spreadsheet  allows saving into csv file


Square root of $2$ is $\sqrt{2}$ and $n$th root of $2$ is $\sqrt[n]{2}$

\cite{suggestivename:someother} 

\include{chap1}

\include{chap2}

$\mathbb{Z}$ is the set of integers $\alpha$ 

$\int$

$\sqrtntwo$

$\sqrtntwo$

$\sqrtntwo$

$\sqrtntwo$

The total of $1$ to $n$ is $\sum_1^n$ is
\[
\sum_1^n 
\int_1^n 
\]
Here are some fractions  
Mathmode $mathmode figure $  {\it figure e f g h}
\[
\mbox{For all $x$ we try} math~mode
\]

  % Latex questions: listings color in python
  % Figure: left/right/center   or page top, bottom? 

$5^{p^{7^{4_6}}}$

\[
\left[ \begin{array}{rrr}
4 & 4  &5 \\ 6 & -4 & p \end{array} \right\} 
\]

We have some text and some formula $E = m c^2$ 
\[
E =  m c ^ 2 
\]

\begin{equation} \label{equation:einstein}
E = m c^2 
\end{equation}
This is formula \ref{equation:einstein}.

\vspace*{2cm}

\begin{table}
\caption{My favorite table} \label{table:ten}
\begin{longtable}{|l|c|p{4cm}|}
\hline  \\ % [-3mm]
Header row's first column & 
Header row's second column & 
Header row's third column \\
\hline \hline
First column & second column & third column \\
First column second row & second column & third column \\
First column second row & second column & third column \\
First column second row & second column & third column \\
First column second row & second column & third column \\
First column second row & second column & third column \\
First column second row & second column & third column \\
First column second row & second column & third column \\
First column second row & second column & third column \\
First column second row & second column & third column \\
First column second row & second column & third column \\
First column second row & second column & third column \\
First column second row & second column & third column \\
First column second row & second column & third column \\
First column second row & second column & third column \\
First column second row & second column & third column \\
First column second row & second column & third column \\
First column second row & second column & third column \\
First column second row & second column & third column \\
First column second row & second column & third column \\
First column second row & second column & third column \\
First column second row & second column & third column \\
First column second row & second column & third column \\
First column second row & second column & third column \\
First column second row & second column & third column \\
First column second row & second column & third column \\
First column second row & second column & third column \\
First column second row & second column & third column \\
First column second row & second column & third column \\
First column second row & second column & third column \\
First column second row & second column & third column \\
First column second row & second column & third column \\
First column second row & second column & third column \\
First column second row & second column & third column \\
First column second row & second column & third column \\
First column & second column & third column \\ \hline
\end{longtable}
\end{table}

\section{Test section}

\chapter{Second chapter}

\section{Another Test section}

\begin{figure}[htb!] 
\includegraphics[angle=45, width=6cm]{test-image.png}
\caption{This is our figure.} 
\label{ourfigure}
\end{figure}

Things are explained in the figure (Figure \ref{ourfigure} )

\lstset{language=Python, keywordstyle=\color{red}}
\begin{lstlisting}

for i = 1:5
   print i

\end{lstlisting}

\vspace{5cm}

\verb*&This on  e wi ll come verbatim&

\noindent
This one will not come verbatim



``For double quotes."

For {\em emphasis}

Another way
For \emph{emphasis}

For bold \textbf{bold}

Or similarly {\bf this one is bold}

\begin{flushright}
This part will be flushed right
\end{flushright}



\begin{enumerate}
\item This part will be centered

\item This part also will be centered\footnote{Text for footnote
\label{footnote-my-name}. No section labels here! }

\item This part also \texttt{will be centered}
\item More items. This was explained in footnote number 
\ref{footnote-my-name} 
\end{enumerate}













\section{New section added section}

The first section got re-introduced now.

\section{First section} \label{our:name:section1}

Here comes the content.
IT was indeed discussed right here.

Here comes the content.

\section*{SEcond section}

This was already discussed in Section \ref{our:name:section1}

There are special symbols:

\$ \% \& \# \\ does not give a backslash

\section{Third section}

\textbackslash ~ gives

$ math mode$ 

%  means comment


$\backslash$ gives

\begin{thebibliography}{10}

\bibitem[Al1]{suggestivename:FOSSEE} Author name, initials, Proceedings of FOSSEE conference


\bibitem[Al2]{suggestivename:someother} Author different, initials, Proceedings of another conference
\end{thebibliography}

\end{document}

This is ignored by latex
